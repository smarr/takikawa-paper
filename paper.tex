\documentclass[sigplan,10pt,review,screen]{acmart}\settopmatter{printfolios=true}

\acmConference[VMIL'19]{ACM SIGPLAN International Workshop on Virtual Machines and Intermediate Languages}{October 22, 2019}{Athens, Greece}
\acmYear{2019}
\acmISBN{978-x-xxxx-xxxx-x/YY/MM}
\acmDOI{10.1145/nnnnnnn.nnnnnnn}
\startPage{1}

\setcopyright{none}
\bibliographystyle{ACM-Reference-Format}

\begin{document}
\title{How slow do my transient type checks go?}

\author{Isaac Oscar Gariano}
\affiliation{
  \department{Engineering and Computer Science} % \department is recommended
  \institution{Victoria University of Wellington}
  \country{New Zealand}
}
\email{Isaac@ecs.vuw.ac.nz} % \email is recommended

\author{Richard Roberts}
\affiliation{
  \department{Computational Media Innovation Centre} % \department is recommended
  \institution{Victoria University of Wellington}
  \country{New Zealand}
}
\email{rykardo.r@gmail.com} % \email is recommended


\author{Stefan Marr}
\affiliation{
  \department{School of Computing} % \department is recommended
  \institution{University of Kent}
  \country{s.marr@kent.ac.uk}
}
\email{mwh@ecs.vuw.ac.nz} % \email is recommended


\author{Michael Homer}
\affiliation{
  \department{Engineering and Computer Science} % \department is recommended
  \institution{Victoria University of Wellington}
  \country{New Zealand}
}
\email{mwh@ecs.vuw.ac.nz} % \email is recommended



\author{James Noble}
\orcid{0000-0001-9036-5692}             %% \orcid is optional
\affiliation{
  \department{Engineering and Computer Science} % \department is recommended
  \institution{Victoria University of Wellington}
  \country{New Zealand}
}
\email{kjx@ecs.vuw.ac.nz} % \email is recommended




\begin{abstract}
Just-in-time compilation and optimisation in virtual machines can
eliminate much of the overhead of transient typechecks.  Unfortunately
the improvement is not uniform: while most typechecks can be optimised
away, some typechecks will will significantly decrease a program's
performance.  In this paper we investigate how benchmark results can be
indicate which typechecks cause the most problems. Programmers could
use these techniques to optimise their programs by removing
typechecks, and VM engineers to identify opportunities for further
optimisations.   
\end{abstract}


%% Generate at 'http://dl.acm.org/ccs/ccs.cfm'.
%%KJX XML IS BROKEN!!!
\begin{CCSXML}
\end{CCSXML}

\ccsdesc[500]{Software and its engineering~Just-in-time compilers}
\ccsdesc[300]{Software and its engineering~Object oriented languages}
\ccsdesc[300]{Software and its engineering~Interpreters}

\keywords{dynamic type checking, gradual types, optional types, Grace,
Moth, object-oriented programming}
\maketitle


\section{Introduction}

intro (short \& modified)

\section{Background}

BG: moth \& transient typing, probably has to inclde takikawa

\section{Evaluating Transient Typechecks}

test protocol

\section{Results}

core results - i.e. looking at one scatterplot per annotation - the original version
   - is there a good benchmark that is small enough to fit on one page. and shows interesting results?


\section{Individual Annotations}
     
   trying-to-be-cleverer results
  - single annotation overheads
  - using them go back to scatterplots (i.e. what you do in the latest graphs) 

\section{Discussion and Conclusion}

  discuccion / future work / conclusion (statistics, potentially doing this for other things)


\end{document}
